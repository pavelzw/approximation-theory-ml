% BibTeX setup
\usepackage[backend=bibtex, bibstyle=alphabetic, citestyle=alphabetic]{biblatex}
\bibliography{references}

% \usepackage[english]{babel} % English
\usepackage[ngerman]{babel} % German

\usepackage{enumitem} % Package for handling of lists
\setlist{noitemsep}   % No separating whitespace between list items.

% Standard packages for math-related things.
\usepackage{amsmath}
\usepackage{amssymb}
\usepackage{amsthm}

\usepackage{graphicx} % to include graphics with \includegraphics[options]{imagefile}

% Hyperref is great, but sometimes there are problems if packages are loaded before or after hyperref. Check the documentation.
\usepackage{hyperref} % For PDF links, toc, etc.

% Formatting of paragraphs
\parindent 0cm                     % No intendation at the beginning of a paragraph
\parskip1.5ex plus0.5ex minus0.5ex % Vertical space between two paragraphs


%%%%%%%%%%%%%%%%%%%%%%%%%%%%%%%%%%%%%%%%%%%%%%%%%%%%%%%%%%%%%%%%%%%%%%%%%%%%%%%%
% Theorem and style definitions (for amsthm). For details see
% https://en.wikibooks.org/wiki/LaTeX/Theorems
%%%%%%%%%%%%%%%%%%%%%%%%%%%%%%%%%%%%%%%%%%%%%%%%%%%%%%%%%%%%%%%%%%%%%%%%%%%%%%%%

\theoremstyle{plain} % Usual style for theorems, etc.

% All numbered with the same counter (theorem).
% \newtheorem{theorem}{Theorem}[section] % This uses numbering SECTION.COUNT instead of COUNT. Useful in longer documents.
\newtheorem{theorem}{Theorem} % Number linearly (no SECTION prefix).
% Usage: \newtheorem{environmentname}[counter]{displayedtext}
\newtheorem{proposition}[theorem]{Proposition}
\newtheorem{lemma}[theorem]{Lemma}
\newtheorem{corollary}[theorem]{Corollary}
\newtheorem*{lemma*}{Lemma} % not numbered.

\theoremstyle{definition} % Usual style definitions.
\newtheorem{definition}[theorem]{Definition}

\theoremstyle{remark} % Usual style definitions.
\newtheorem{remark}[theorem]{Remark}


%%%%%%%%%%%%%%%%%%%%%%%%%%%%%%%%%%%%%%%%%%%%%%%%%%%%%%%%%%%%%%%%%%%%%%%%%%%%%%%%
% Some commands
%%%%%%%%%%%%%%%%%%%%%%%%%%%%%%%%%%%%%%%%%%%%%%%%%%%%%%%%%%%%%%%%%%%%%%%%%%%%%%%%

\newcommand{\IN}{\mathbb{N}} % blackboard bold N for natural numbers
\newcommand{\IR}{\mathbb{R}} % blackboard bold R for real numbers
\newcommand{\IZ}{\mathbb{Z}} % blackboard bold Z for integers
\newcommand{\IF}{\mathbb{F}} % blackboard bold F for fields
\newcommand{\IG}{\mathbb{G}} % blackboard bold G for groups

% feel free to add more commands here

\newcommand{\set}[1]{\{#1\}}