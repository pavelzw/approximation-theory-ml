\documentclass[11pt]{scrartcl}

%%%%%%%%%%%%%%%%%%%%%%%%%%%%%%%%%%%%%%%%%%%%%%%%%%%%%%%%%%%%%%%%%%%%%%%%%%%%%%%%
% Make this compatible with pdflatex and lualatex, xelatex.
%%%%%%%%%%%%%%%%%%%%%%%%%%%%%%%%%%%%%%%%%%%%%%%%%%%%%%%%%%%%%%%%%%%%%%%%%%%%%%%%

\usepackage{ifxetex,ifluatex}
\newif\ifxetexorluatex
\ifxetex
  \xetexorluatextrue
\else
  \ifluatex
    \xetexorluatextrue
  \else
    \xetexorluatexfalse
  \fi
\fi

\ifxetexorluatex
  \usepackage{fontspec}
  \usepackage{xltxtra} % Not that important nowa
  % Font packages, and other xetex specific configuration here.
\else
  % Use utf8 as encoding.
  \usepackage[utf8]{inputenc}
  \usepackage[T1]{fontenc}
  % font packages
  \usepackage{lmodern} % Better than default computer modern?
\fi

%%%%%%%%%%%%%%%%%%%%%%%%%%%%%%%%%%%%%%%%%%%%%%%%%%%%%%%%%%%%%%%%%%%%%%%%%%%%%%%%
% End of pdflatex/xelatex/lualatex specific configuration
%%%%%%%%%%%%%%%%%%%%%%%%%%%%%%%%%%%%%%%%%%%%%%%%%%%%%%%%%%%%%%%%%%%%%%%%%%%%%%%%

\usepackage[hmargin=2cm,vmargin=2.5cm]{geometry} % Set small margins for handout

% BibTeX setup
\usepackage[backend=bibtex, bibstyle=alphabetic, citestyle=alphabetic]{biblatex}
\bibliography{references}

% \usepackage[english]{babel} % English
\usepackage[ngerman]{babel} % German

\usepackage{enumitem} % Package for handling of lists
\setlist{noitemsep}   % No separating whitespace between list items.

% Standard packages for math-related things.
\usepackage{amsmath}
\usepackage{amssymb}
\usepackage{amsthm}

\usepackage{mathtools}

\usepackage{graphicx} % to include graphics with \includegraphics[options]{imagefile}

\usepackage{tikz}
\usepackage{pgfplots}

% Hyperref is great, but sometimes there are problems if packages are loaded before or after hyperref. Check the documentation.
\usepackage{hyperref} % For PDF links, toc, etc.

% Formatting of paragraphs
\parindent 0cm                     % No intendation at the beginning of a paragraph
\parskip1.5ex plus0.5ex minus0.5ex % Vertical space between two paragraphs


%%%%%%%%%%%%%%%%%%%%%%%%%%%%%%%%%%%%%%%%%%%%%%%%%%%%%%%%%%%%%%%%%%%%%%%%%%%%%%%%
% Theorem and style definitions (for amsthm). For details see
% https://en.wikibooks.org/wiki/LaTeX/Theorems
%%%%%%%%%%%%%%%%%%%%%%%%%%%%%%%%%%%%%%%%%%%%%%%%%%%%%%%%%%%%%%%%%%%%%%%%%%%%%%%%

\theoremstyle{plain} % Usual style for theorems, etc.

% All numbered with the same counter (theorem).
% \newtheorem{theorem}{Theorem}[section] % This uses numbering SECTION.COUNT instead of COUNT. Useful in longer documents.
\newtheorem{theorem}{Satz} % Number linearly (no SECTION prefix).
% Usage: \newtheorem{environmentname}[counter]{displayedtext}
\newtheorem{proposition}[theorem]{Proposition}
\newtheorem{lemma}[theorem]{Lemma}
\newtheorem{corollary}[theorem]{Korollar}
\newtheorem*{lemma*}{Lemma} % not numbered.

\theoremstyle{definition} % Usual style definitions.
\newtheorem{definition}[theorem]{Definition}

\theoremstyle{remark} % Usual style definitions.
\newtheorem{remark}[theorem]{Remark}


%%%%%%%%%%%%%%%%%%%%%%%%%%%%%%%%%%%%%%%%%%%%%%%%%%%%%%%%%%%%%%%%%%%%%%%%%%%%%%%%
% Some commands
%%%%%%%%%%%%%%%%%%%%%%%%%%%%%%%%%%%%%%%%%%%%%%%%%%%%%%%%%%%%%%%%%%%%%%%%%%%%%%%%

\newcommand{\IN}{\mathbb{N}} % blackboard bold N for natural numbers
\newcommand{\IR}{\mathbb{R}} % blackboard bold R for real numbers
\newcommand{\IZ}{\mathbb{Z}} % blackboard bold Z for integers
\newcommand{\IF}{\mathbb{F}} % blackboard bold F for fields
\newcommand{\IG}{\mathbb{G}} % blackboard bold G for groups

% feel free to add more commands here

\newcommand{\set}[1]{\{#1\}}

%%%%%%%%%%%%%%%%%%%%%%%%%%%%%%%%%%%%%%%%%%%%%%%%%%%%%%%%%%%%%%%%%%%%%%%%%%%%%%%%
% Content begins here
%%%%%%%%%%%%%%%%%%%%%%%%%%%%%%%%%%%%%%%%%%%%%%%%%%%%%%%%%%%%%%%%%%%%%%%%%%%%%%%%

\def\Semester{WS 21/22}
%\def\Semester{SS 2021}
\def\Seminar{Mathematische Aspekte des maschinellen Lernens}
\def\Title{Approximationstheorie} 
\def\Author{Pavel Zwerschke} 
\def\Date{19.11.2021} % The date of your presentation

% Font
\renewcommand{\familydefault}{\sfdefault}

\newcommand{\N}{\mathbb{N}} % blackboard bold N for natural numbers
\newcommand{\R}{\mathbb{R}} % blackboard bold R for real numbers
\newcommand{\Z}{\mathbb{Z}} % blackboard bold Z for integers

% Start of document
\begin{document}

\vspace*{-2cm}
\begin{minipage}{3cm}
%	\includegraphics[width=3cm]{./logos/kit-en.pdf} % English logo
	\includegraphics[width=3cm]{./logos/kit-de.pdf} % German logo
\end{minipage}\hspace*{0.2cm}~
\begin{minipage}{14cm}{
		\sffamily \Large{\Seminar} \hfill \Date \\ 
		\Author \hfill \Semester
}\end{minipage}
\vspace{-0.5cm}
\begin{center}
	\huge \sffamily \Title
\end{center}
\vspace{-1cm}

\section{Voraussetzungen}
\label{sec:prerequisites}

\begin{definition}[ReLU Netzwerk]
	Sei \(L \in \N\) und \(N_0, N_1, \ldots, N_L \in \N\). Ein 
	\textit{ReLU neuronales Netz} \(\Phi\) ist eine Abbildung 
	\(\Phi: \R^{N_0} \rightarrow \R^{N_L}\), die durch 
	\[ \Phi = \begin{cases}
		W_1, & L = 1, \\
		W_L \circ \rho \circ \cdots \circ \rho \circ W_1, & L \geq 2
	\end{cases} \]
	gegeben ist. Hierbei ist für \(l\in \set{1,\ldots, L}\) \(W_l : \R^{N_{l-1}} \rightarrow \R^{N_l}, W_l(x) = A_l x + b_l\) 
	die jeweilige affine Transformation und \(\rho: \R \rightarrow \R\), \(\rho(x) := \max\set{0, x}\) 
	die ReLU-Funktion (komponentenweise).

	Die Menge aller ReLU Netzwerke mit Input-Dimension \(N_0 = d\) und Output-Dimension \(N_L = d'\) 
	definieren wir als \(\mathcal{N}_{d,d'}\).
\end{definition}

\begin{definition}[Eigenschaften eines ReLU Netzwerks]\leavevmode
	\begin{description}
		\item[Konnektivität] \(\mathcal{M}(\Phi)\) ist die Anzahl der nicht null Einträge in \(A_1, \ldots, A_L\) 
		sowie \(b_1, \ldots, b_L\),
		\item[Tiefe] \(\mathcal{L}(\Phi) := L\),
		\item[Breite] \(\mathcal{W}(\Phi) := \max_{l=0,\ldots, L} N_l\),
		\item[Gewicht] \(\mathcal{B}(\Phi) := \max_{l=1,\ldots, L} \max\set{||A_l ||_\infty, ||b_l||_\infty}\).
	\end{description}
\end{definition}

\begin{lemma}
	Seien \(d_1\), \(d_2\), \(d_3 \in \N\), \(\Phi_1 \in \mathcal{N}_{d_1, d_2}\). Dann existiert ein 
	Netzwerk \(\Psi \in \mathcal{N}_{d_1, d_3}\) mit \(\mathcal{L}(\Psi) = \mathcal{L}(\Phi_1) + \mathcal{L}(\Phi_2)\), 
	\(\mathcal{M}(\Psi) \leq 2 \mathcal{M}(\Phi_1) + 2\mathcal{M}(\Phi_2)\), \(\mathcal{W}(\Psi) \leq 
	\max\set{2d_2, \mathcal{W}(\Phi_1), \mathcal{W}(\Phi_2)}\), \(\mathcal{B}(\Psi) = \max\set{\mathcal{B}(\Phi_1), \mathcal{B}(\Phi_2)}\) 
	sowie
	\[ \Psi(x) = (\Phi_2 \circ \Phi_1)(x) = \Phi_2(\Phi_1(x)) \;\forall x\in \R^{d_1}. \]
\end{lemma}

\begin{lemma}
	Sei \(n, L \in\N\) und für \(i\in \set{1,\ldots, n}\) seien \(d_i, d_i' \in \N\) und \(\Phi_i \in \mathcal{N}_{d_i, d_i'}\) 
	mit \(\mathcal{L}(\Phi_i) = L\). Dann existiert ein Netzwerk \(\Psi \in \mathcal{N}_{\sum_{i=1}^n d_i, \sum_{i=1}^n d_i'}\) 
	mit \(\mathcal{L}(\Psi) = L\), \(\mathcal{M}(\Psi) = \sum_{i=1}^n \mathcal{M}(\Phi_i)\), 
	\(\mathcal{W}(\Psi) = \sum_{i=1}^n \mathcal{W}(\Phi_i)\), \(\mathcal{B}(\Psi) = \max_i \mathcal{B}(\Phi_i)\) sowie 
	\[ \Psi(x) = (\Phi_1(x_1), \ldots, \Phi_n(x_n)) \in \R^{\sum_{i=1}^n d_i'} \]
	für \(x = (x_1, \ldots, x_n) \in \R^{\sum_{i=1}^n d_i}\) mit \(x_i \in \R^{d_i}\), \(i\in \set{1,\ldots, n}\).
\end{lemma}

\begin{lemma}
	Sei \(n, L, d' \in\N\) und für \(i\in \set{1,\ldots, n}\) seien \(d_i \in \N\), \(a_i \in \R\) und \(\Phi_i \in \mathcal{N}_{d_i, d'}\) 
	mit \(\mathcal{L}(\Phi_i) = L\). Dann existiert ein Netzwerk \(\Psi \in \mathcal{N}_{\sum_{i=1}^n d_i, d'}\) 
	mit \(\mathcal{L}(\Psi) = L\), \(\mathcal{M}(\Psi) \leq \sum_{i=1}^n \mathcal{M}(\Phi_i)\), 
	\(\mathcal{W}(\Psi) = \sum_{i=1}^n \mathcal{W}(\Phi_i)\), \(\mathcal{B}(\Psi) = \max_i \set{|a_i| \mathcal{B}(\Phi_i)}\) sowie 
	\[ \Psi(x) = \sum_{i=1}^n a_i \Phi_i(x_i) \in \R^{d'} \]
	für \(x = (x_1, \ldots, x_n) \in \R^{\sum_{i=1}^n d_i}\) mit \(x_i \in \R^{d_i}\), \(i\in \set{1,\ldots, n}\).
\end{lemma}

\section{Approximationstheorie}

\begin{definition}[Sawtooth Konstruktion]
	\[g(x) \coloneqq 2\rho(x) - 4 \rho(x - \frac{1}{2}) + 2\rho(x-1) = \begin{cases}
        2x, & 0\leq x < \frac{1}{2}, \\ 
        2(1-x), & \frac{1}{2} \leq x \leq 1, \\
        0, & \text{sonst.}
    \end{cases}\]
    \[g_s \coloneqq \underbrace{g \circ \cdots \circ g}_s \]
\end{definition}
\begin{center}
	\begin{tikzpicture}[scale=2]
	   \draw[->] (-0.25, 0) -- (1.25, 0) node[right] {$x$};
	   \draw[->] (0, -0.25) -- (0, 1.25) node[above] {$y$};
	   \draw[domain=0:0.5, smooth, variable=\x, blue] plot ({\x}, {2*\x});
	   \draw[domain=0.5:1, smooth, variable=\x, blue]  plot ({\x}, {2*(1-\x)});
	   \node at (0.5, 1.5) {\(g_1\)};
	\end{tikzpicture}
	\begin{tikzpicture}[scale=2]
	   \draw[->] (-0.25, 0) -- (1.25, 0) node[right] {$x$};
	   \draw[->] (0, -0.25) -- (0, 1.25) node[above] {$y$};
	   \draw[domain=0:0.25, smooth, variable=\x, blue] plot ({\x}, {4*\x});
	   \draw[domain=0.25:0.5, smooth, variable=\x, blue]  plot ({\x}, {4*(0.5-\x)});
	   \draw[domain=0.5:0.75, smooth, variable=\x, blue]  plot ({\x}, {4*(\x-0.5)});
	   \draw[domain=0.75:1, smooth, variable=\x, blue]  plot ({\x}, {4*(1-\x)});
	   \node at (0.5, 1.5) {\(g_2\)};
	\end{tikzpicture}
	\begin{tikzpicture}[scale=2]
	   \draw[->] (-0.25, 0) -- (1.25, 0) node[right] {$x$};
	   \draw[->] (0, -0.25) -- (0, 1.25) node[above] {$y$};
	   \draw[domain=0:0.125, smooth, variable=\x, blue] plot ({\x}, {8*\x});
	   \draw[domain=0.125:0.25, smooth, variable=\x, blue]  plot ({\x}, {8*(0.25-\x)});
	   \draw[domain=0.25:0.375, smooth, variable=\x, blue]  plot ({\x}, {8*(\x-0.25)});
	   \draw[domain=0.375:0.5, smooth, variable=\x, blue]  plot ({\x}, {8*(0.5-\x)});
	   \draw[domain=0.375:0.5, smooth, variable=\x, blue]  plot ({\x}, {8*(0.5-\x)});
	   \draw[domain=0.5:0.625, smooth, variable=\x, blue]  plot ({\x}, {8*(\x-0.5)});
	   \draw[domain=0.625:0.75, smooth, variable=\x, blue]  plot ({\x}, {8*(0.75-\x)});
	   \draw[domain=0.75:0.875, smooth, variable=\x, blue]  plot ({\x}, {8*(\x-0.75)});
	   \draw[domain=0.875:1, smooth, variable=\x, blue]  plot ({\x}, {8*(1-\x)});
	   \node at (0.5, 1.5) {\(g_3\)};
	\end{tikzpicture}
\end{center}

\begin{lemma}
	Sei \(s\in\N\), \(k \in \set{0,1,\ldots, 2^{s-1}-1}\). Dann gilt \(g(2^{s-1} \cdot - k)\) 
    hat Träger \([\frac{k}{2^{s-1}}, \frac{k+1}{2^{s-1}}]\), weiter gilt 
    \[ g_s(x) = \sum_{k=0}^{2^{s-1}-1} g(2^{s-1}x - k), x\in [0,1] \]
    sowie 
    \[ g_s(\frac{k}{2^{s-1}}+x) = g_s(\frac{k+1}{2^{2-1}} - x), x \in [0, \frac{1}{2^{s-1}}]. \]
\end{lemma}

\begin{proposition} % iii.2
	Es existiert eine Konstante \(C>0\), sodass für alle \(\varepsilon \in (0,1/2)\) 
	ein Netzwerk \(\Phi_\varepsilon \in \mathcal{N}_{1,1}\) mit 
	\(\mathcal{L}(\Phi_\varepsilon) \leq C\log(\varepsilon^{-1})\), 
	\(\mathcal{W}(\Phi_\varepsilon) = 3\), \(\mathcal{B}(\Phi_\varepsilon) = 1\), 
	\(\Phi_\varepsilon(0) = 0\) existiert mit 
	\[ ||\Phi_\varepsilon(x) - x^2 ||_{L^\infty([0,1])} \leq \varepsilon. \]
\end{proposition}

\begin{proposition} % iii.3
	Es existiert eine Konstante \(C>0\), sodass für alle \(D\in \R_+\) und \(\varepsilon \in (0, 1/2)\) 
	ein Netzwerk \(\Phi_{D,\varepsilon} \in \mathcal{N}_{2,1}\) existiert mit 
	\(\mathcal{L}(\Phi_{D, \varepsilon}) \leq C (\log(\lceil D \rceil) + \log(\varepsilon^{-1})) \), 
	\(\mathcal{W}(\Phi_{D, \varepsilon}) \leq 5\), \(\mathcal{B}(\Phi_{D, \varepsilon}) = 1\) sowie 
	\(\Phi_{D,\varepsilon}(0,x) = \Phi_{D,\varepsilon}(x,0) = 0 \;\forall x\in \R\) und 
	\[ ||\Phi_{D,\varepsilon}(x,y) - xy||_{L^\infty([-D,D]^2)} \leq \varepsilon. \]
\end{proposition}

\begin{proposition} % iii.5
	\newcommand{\Phia}{\Phi_{a,D,\varepsilon}}
	Es existiert eine Konstante \(C>0\), sodass für alle \(m\in \N\), \(a = (a_i)_{i=0}^m \in \R^{m+1}\), 
	\(D\in \R_+\) und \(\varepsilon \in (0,1/2)\) ein Netzwerk \(\Phia \in \mathcal{N}_{1,1}\) 
	existiert mit \(\mathcal{L}(\Phia) \leq C m (\log(\varepsilon^{-1}) + m\log(\lceil D \rceil) + \log(m) + \log(\lceil ||a||_\infty \rceil))\), 
	\(\mathcal{W}(\Phia) \leq 9\), \(\mathcal{B}(\Phia) \leq 1\) sowie 
	\[ \left|\left|\Phia(x) - \sum_{i=0}^m a_i x^i \right|\right|_{L^\infty([-D,D])} \leq \varepsilon. \]
\end{proposition}

\begin{lemma} % iii.7
	Betrachte die Menge der Funktionen 
	\[ \mathcal{S}_{[-1,1]} \coloneqq \left\{ f \in C^\infty([-1,1], \R): ||f^{(n)}||_{L^\infty([-1,1])} \leq n! \;\forall n \in \N_0 \right\}. \]
	Dann existiert eine Konstante \(C>0\), sodass für alle \(f\in \mathcal{S}_{[-1,1]}\) und \(\varepsilon\in (0,1/2)\) 
	ein Netz \(\Psi_{f,\varepsilon} \in \mathcal{N}_{1,1}\) existiert mit 
	\( \mathcal{L}(\Psi_{f,\varepsilon}) \leq C(\log(\varepsilon^{-1}))^2 \), 
	\(\mathcal{W}(\Psi_{f,\varepsilon}) \leq 9\), \(\mathcal{B}(\Psi_{f,\varepsilon}) \leq 1\) 
	sowie 
	\[ ||\Psi_{f,\varepsilon} - f||_{L^\infty([-1,1])} \leq \varepsilon. \]
\end{lemma}

\begin{theorem} % iii.8
	\newcommand{\Psia}{\Psi_{a,D,\varepsilon}}
	Es existiert eine Konstante \(C>0\), sodass für alle \(a,D\in \R_+\), \(\varepsilon \in (0,1/2)\) 
	ein Netzwerk \(\Psia \in \mathcal{N}_{1,1}\) mit \(\mathcal{L}(\Psia) \leq C((\log(\varepsilon^{-1}))^2 + \log(\lceil aD\rceil))\), 
	\(\mathcal{W}(\Psia) \leq 9\), \(\mathcal{B}(\Psia) \leq 1\) sowie 
	\[ ||\Psia - \cos(a x) ||_{L^{\infty}([-D,D])} \leq \varepsilon. \]
\end{theorem}

%%%%%%%%%%%%%%%%%%%%%%%%%%%%%%%%%%%%%%%%%%%%%%%%%%%%%%%%%%%%%%%%%%%%%%%%%%%%%%%%
% Content ends here
%%%%%%%%%%%%%%%%%%%%%%%%%%%%%%%%%%%%%%%%%%%%%%%%%%%%%%%%%%%%%%%%%%%%%%%%%%%%%%%%
\nocite{Grohs2019}
\printbibliography{}
\end{document}